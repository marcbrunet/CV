% -- Encoding UTF-8 without BOM
% -- XeLaTeX => PDF (BIBER)

\documentclass[]{cv-style}          % Add 'print' as an option into the square bracket to remove colours from this template for printing. 
                                    % Add 'espanol' as an option into the square bracket to change the date format of the Last Updated Text

\sethyphenation[variant=british]{english}{} % Add words between the {} to avoid them to be cut 
\usepackage{caption}
\usepackage{subcaption}
\begin{document}

\header{Marc}{Brunet}           % Your name
\lastupdated

%----------------------------------------------------------------------------------------
%	SIDEBAR SECTION  -- In the aside, each new line forces a line break
%----------------------------------------------------------------------------------------
\vspace{0.5cm}
\begin{aside}
%
\section{contacto}
Calders 38
Navarcles, Barcelona 08270
~
+34 651 06 15 55
~
marcbrunet80@gmail.com
%
%\section{languages}
%English mother tongue
%Spanish fluency
%
\section{programming}
\includegraphics[width=5em]{programing/python.png}\smallskip
\includegraphics[width=5em]{programing/golang.png}\smallskip
\includegraphics[width=5em]{programing/Latex.png}\smallskip
\includegraphics[width=6em]{programing/java.png}
%
\section{DevOps Tols}
\includegraphics[width=5em]{devops/jenkins.png}\smallskip
\includegraphics[width=5em]{devops/docker.png}\smallskip
\includegraphics[width=5em]{devops/ansibl.png}\smallskip
\includegraphics[width=5em]{devops/vagrant.png}
\end{aside}

%----------------------------------------------------------------------------------------
%	SKILLS SECTION
%----------------------------------------------------------------------------------------

%\section{skills}
%  \vspace{-0.2cm}

%Skill 1, skill 2, skill 3, skill 4, skill 5.

%----------------------------------------------------------------------------------------
%	WORK EXPERIENCE SECTION
%----------------------------------------------------------------------------------------

\section{experiencia}


\begin{entrylist}
%------------------------------------------------
%\entry
%  {2014--Now}
%  {COMPANY 3}
%  {City, Country}
%  {\jobtitle{Job Title}\\
%  Job description. Job description. Job description. Job description. Job description. Job description. Job description. Job description. Job description. Job description. Job description.}
%------------------------------------------------
\entry
  {02/2018--11/2018}
  {ROCHE}
  {Sant Cugat del Vallès, Barcelona}
  {\jobtitle{DevOps}\\
  Tareas de administración de la infraestructura y mantenimiento del datacenter local, convirtiendo posesos en maquina virtuales en Docker y automatizando tares de mantenimiento de maquinas windows y linux, automatización del posesos de entrega continua con jenkigns, creación de los entornos personalizados con packer y ansible
  }
  
%------------------------------------------------
\entry
  {06/2017--06/2014}
  {Daitec}
  {Sampedor, Barcelona}
  {\jobtitle{Programador de maquinaria industrial}\\
  Feines de programació en oficina, remotament i a casa el client, programació de PLC per a maquinària industrials (Siemens), xarxes (VPN, Firewall), introducció al .NET, programador panells web amb PHP}
%------------------------------------------------


\end{entrylist}

%----------------------------------------------------------------------------------------
%	EDUCATION SECTION
%----------------------------------------------------------------------------------------

\section{education}

\begin{entrylist}
%------------------------------------------------
%\entry
%{2010--2011}
%{M.Sc. {\normalfont in Economics [Grade]}}
%{University}
%{\vspace{-0.3cm}}
%------------------------------------------------
\entry
{2018--now}
{Postgrau {\normalfont Administració de Sistemes, DevOps i Cloud}}
{UOC}
{Trabajo final de carrera en el control de un vehículo eléctrico, diseño y construcción de una moto eléctrica para la competición de motostudent.}


%------------------------------------------------
\entry
{2014--2018}
{Enginyeria/Grau {\normalfont Sistemes TIC}}
{UPC de Manresa. EPSEM}
{Trabajo final de carrera en el control de un vehículo eléctrico, diseño y construcción de una moto eléctrica para la competición de motostudent.}

%------------------------------------------------
\entry
{2012--2014}
{CFGS/FP {\normalfont en Automatització i Robòtica Industrial}}
{Institut Lacetània}
{Participació en les catSkils, qualificació 3r}

%------------------------------------------------
\end{entrylist}

%----------------------------------------------------------------------------------------
%	OTHER QUALIFICATIONS SECTION
%----------------------------------------------------------------------------------------

%\section{other qualifications}

%\begin{entrylist}
%------------------------------------------------
%\entry
%{2013}
%{Qualification}
%{Institution}
%{\vspace{-0.3cm}}
%------------------------------------------------
%\entry
%{2011}
%{Qualification}
%{Institution}
%{\vspace{-0.3cm}}
%------------------------------------------------
%\end{entrylist}

%----------------------------------------------------------------------------------------
%	AWARDS SECTION
%----------------------------------------------------------------------------------------

%\section{awards}

%\begin{entrylist}
%------------------------------------------------
%\entry
%{2014}
%{Award name}
%{Institution}
%{Award description. Award description. Award description. Award description. Award description. Award description. Award description. }
%------------------------------------------------
%\end{entrylist}

%----------------------------------------------------------------------------------------
%	INTERESTS SECTION
%----------------------------------------------------------------------------------------

\section{interessos}
  \vspace{-0.2cm}

\textbf{professional:} estic molt interessat en els entorns Cloud, és a dir en el sistema que s'utilitza per gestionar serveis en local i Cloud mantenir-los en actiu i monotonitzar-los per intentar obtenir millores de rendiment o detectar errors, també amb el tema de la integració continua i testing automatitzat per tal d'obtenir resultats el mes ràpid possible. \smallskip

\textbf{personal:} M'agrada l'electrònica i la impressió en 3 d, des del tema de com es genera el Gcode apartí d'un objecte en 3 d fins a com és gestiona'n el moviment de la màquina per tal d'obtenir la pesa finalitzada, i el monitoratge de la impressió, utilitzant eines com:

\vspace{-0.2cm}
\begin{figure}[!htb]
\minipage{0.20\textwidth}
  \includegraphics[width=5em]{intaresos/arduino.png}
\endminipage\hfill
\minipage{0.20\textwidth}
  \includegraphics[width=5em]{intaresos/raspberry.png}
\endminipage\hfill
\minipage{0.20\textwidth}%
  \includegraphics[width=5em]{intaresos/ubuntu.png}
\endminipage\hfil
\minipage{0.20\textwidth}%
  \includegraphics[width=5em]{intaresos/reprap.png}
\endminipage\hfill
\minipage{0.20\textwidth}%
  \includegraphics[width=5em]{intaresos/octoprint.png}
\endminipage
\end{figure}


%----------------------------------------------------------------------------------------

\end{document}